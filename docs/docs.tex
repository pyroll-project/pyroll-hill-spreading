% Preamble
\documentclass[11pt]{PyRollDocs}
\usepackage{textcomp}

\addbibresource{refs.bib}

% Document
\begin{document}

    \title{The Hill Spreading PyRolL Plugin}
    \author{Christoph Renzing}
    \date{\today}

    \maketitle

    This plugin provides a spreading modelling approach with Hill's formula for flat rolling.


    \section{Model approach}\label{sec:model-approach}

    \subsection{Hill's spread equation}\label{subsec:hill's-spread-equation}

    \textcite{Hill1955} proposed \autoref{eq:hill} for estimation of spreading in flat rolling,
    $h$ and $b$ are height and width of the workpiece with the indices 0 and 1 denoting the incoming respectively the outgoing profile.

    \begin{equation}
        \beta = \frac{b_1}{b_0} = \frac{h_0}{h_1} ^{w}
        \label{eq:hill}
    \end{equation}
    

    $w$ is the spread exponent, by \textcite{Hill1955} is given in \autoref{eq:exponent}, where $R$ is the roll radius.

    \begin{equation}
        w = 0.5 \exp \left( - \frac{b_0}{2 \sqrt{R \Delta h}} \right)
        \label{eq:exponent}
    \end{equation}

    \section{Usage instructions}\label{sec:usage-instructions}

    The plugin can be loaded under the name \texttt{pyroll\_hill\_spreading}.

    An implementation of the \lstinline{spread} hook on \lstinline{RollPass} is provided,
    calculating the spread using the equivalent rectangle approach and hill's model.

    Base implementations of them is provided, so it should work out of the box.
    For \lstinline{hill_exponent} the equation~\ref{eq:exponent} is implemented.
    Provide your own hook implementations or set attributes on the \lstinline{RollPass} instances to alter the spreading behavior.

    \printbibliography

\end{document}